\section{Introduction}
Single-cell RNA sequencing (scRNA-seq) is a powerful tool that is beginning to make important contributions to diverse research areas such as
development~\cite{Semrau2017},  autoimmunity~\cite{Gaublomme2015}, and cancer~\cite{Patel2014}.
Interpreting scRNA-seq remains challenging, however, as the data is confounded by nuisance factors such as variation in capture efficiency and sequencing depth~\cite{vallejos2017normalizing}, amplification bias, batch effects~\cite{shaham2017removal} and transcriptional noise~\cite{wagner2016revealing}. To avoid mistaking nuisance variation for relevant biological diversity, one must therefore account for measurement bias and uncertainty, especially due to the highly abundant false negatives or ``dropout'' events~\cite{kharchenko2014bayesian}.

The challenge of modeling bias and uncertainty in single-cell data has been explored in several recent studies. A common theme in these studies is treating each data point (cell $\times$ gene) as a random variable and fitting a parametric statistical model to this variable. Most existing models are a mixture of an ``expression'' component, which is usually a negative binomial (e.g., ZINB-WaVE~\cite{zinbwave}) or log normal (e.g., BISCUIT~\cite{biscuit}, and a zero (or low expression) component. The parameters of the model are determined by a combination of cell- and gene-level coefficients, and in some cases additional covariates provided as metadata (e.g., biological condition, batch, and cell quality~\cite{zinbwave}). All of these methods can therefore be interpreted as finding a low-dimensional representation of the data which can be used to approximate the parameters of the cell $\times$ gene random variables. Once these models have been fit to the data, they can then in principle be used for various downstream tasks such as normalization (e.g., scaling, correcting batch effects), imputation of missing data, visualization and clustering.

A complementary line of studies focuses on only one of these tasks, often without explicit probabilistic modeling. For instance, SIMLR~\cite{Wang2017} fits a cell-cell similarity matrix, under the assumption that this matrix has a block structure with a fixed number of clusters. The resulting model can be used for clustering and for visualization~\cite{vanDerMaaten2008}. MAGIC~\cite{magic} performs imputation of unobserved (dropout) counts by propagation in a cell-cell similarity graph. Census~\cite{Qiu2017} and SCNorm~\cite{Bacher2017} look for proper scaling factors by explicitly modeling the dependence of gene expression on sequencing depth or spike-in RNA. For differential expression analysis, the most common methods consist of both methods developed for bulk count data (e.g., DESeq2~\cite{deseq2} and edgeR~\cite{edgeR}) as well as methods developed for scRNA-seq data, explicitly accounting for the high dropout rates (e.g., MAST~\cite{mast}). In~\cite{Lin2017}, neural networks are used as function approximators reducing the dimension of single-cell RNA sequencing data; however, this model is a supervised learning method and inherently relies on some labeling of the cells.

While these methods yield insights into biological variation in single-cell data, several significant limitations remain. First, all of the existing distributional modeling methods assume that a low-dimensional manifold underlies the data, and that the mapping from this manifold to the parameters of the model can be captured by a generalized linear model. While the notion of a restricted dimensionality is plausible (reflecting, for example, common regulatory mechanisms among genes or common states among cells), it is difficult to justify the assumption of linearity. Second, different existing methods use their fitted models for different subsets of tasks (e.g., imputation and clustering, but not differential expression~\cite{biscuit}).  Ideally, one would have a single distributional model that would be used for a range of downstream tasks, thus help ensuring consistency and interpretability of the results. Finally, computational scalability is increasingly important. While most existing methods can be applied to no more than tens of thousands of cells, the next generation of tools must scale to the size of recent data sets (commercial~\cite{10x}, or envisioned by consortia such as the Human Cell Atlas~\cite{Regev2017}) that consist of hundreds of thousands of cells or more. 

To address these limitations, we developed a fully probabilistic approach to normalization and downstream analysis of scRNA-seq data, which we refer to as Single-cell Variational Inference (scVI). scVI is based on a hierarchical Bayesian model~\cite{GelmanHill:2007}.
For each cell, a low-dimensional random vector represents its underlying state. Conditional on this state, each observed gene expression level follows a zero-inflated negative binomial (ZINB) distribution, which captures both overdispersion and dropout events~\cite{Grun2014,deseq2,zinbwave}. The parameters of each ZINB distribution are a nonlinear function of the cell state. We implement this function with deep neural network.

We infer cell states under this model using another deep neural network, known as an encoder network~\cite{kingma2013}. It maps the scRNA-seq data to an approximation of the posterior distribution.
The weights of both neural networks are learned from training data through variation inference~\cite{blei2017variational}, which is a computationally efficient alternative to Markov chain Monte Carlo. 

In the remainder of this chapter, we demonstrate the extent to which scVI addresses the current methodological limitations. First, we demonstrate the scalability of scVI to data sets of up to a million cells. Second, we show that, by using non-linear transformations, scVI better fits unseen data (imputation and held-out log-likelihood). Finally, we demonstrate that the model of scVI can be used for a number of tasks, including batch removal and normalization, clustering, dimensionality reduction and visualization, and differential expression. For each of these tasks, we show that scVI compares favorably to the current state-of-the-art methods. scVI is publicly available at
\url{https://github.com/YosefLab/scvi-tools}. An implementation for all of the analysis performed in this chapter is available at \url{https://zenodo.org/badge/latestdoi/125294792}.
