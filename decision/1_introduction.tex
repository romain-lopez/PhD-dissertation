\section{Introduction}

The auto-encoding variational Bayes (AEVB) algorithm relies on neural networks to amortize approximate inference and
performs model selection by maximizing a lower bound on the model evidence~\cite{AEVB, Rezende2014}. In the specific case of variational autoencoders (VAEs), a low-dimensional representation of data is transformed through a learned nonlinear function (another neural network) into the parameters of a conditional likelihood. VAEs achieve impressive performance on pattern-matching tasks like representation/manifold learning and synthetic image generation~\cite{Gulrajani2017}.

Many machine learning applications, however, require decisions, not just compact representations of the data. 
Researchers have accordingly attempted to use VAEs for decision-making applications, including novelty detection in control applications~\cite{Amini2018}, mutation-effect prediction from genomic sequences~\cite{Riesselman2018}, artifact detection~\cite{Ding2018}, and Bayesian hypothesis testing for single-cell RNA sequencing data~\cite{Lopez292037, Xu2019}.
To make decisions based on VAEs, these researchers implicitly appeal to Bayesian decision theory, which counsels taking the action that minimizes expected loss under the posterior distribution~\cite{Fienberg}.

However, for VAEs, the relevant functionals of the posterior cannot be computed exactly.
Instead, after fitting a VAE based on the ELBO, practitioners take one of three approaches to decision-making: i) the variational distribution may be used as a surrogate for the posterior~\cite{Riesselman2018}, ii) the variational distribution may be used as a proposal distribution for importance sampling~\cite{NIPS2018_7699}, or iii) the variational distribution can be ignored once the model is fit, and decisions may be based on an iterative sampling method such as MCMC or annealed importance sampling~\cite{wu2016quantitative}. But will any of these combined procedures (ELBO for model training and one of these methods for approximating posterior expectations) produce good decisions?

They may not, for two reasons.
First, estimates of the relevant expectations of the posterior may be biased and/or may have high variance. The former situation is typical when the variational distribution is substituted for the posterior; the latter is common for importance sampling estimators. By using the variational distribution as a proposal distribution, practitioners aim to get unbiased low-variance estimates of posterior expectations. But this approach often fails. The variational distribution recovered by the VAE, which minimizes the reverse Kullback-Leibler (KL) divergence between the variational distribution and the model posterior, is known to systematically underestimate variance~\cite{wainwright2008graphical,Turner2011}, making it a poor choice for an importance sampling proposal distribution. Alternative inference procedures have been proposed to address this problem. For example, expectation propagation~\cite{minka2013expectation} and CHIVI~\cite{NIPS2017_6866} minimize the forward KL divergence and the $\rchi^2$ divergence, respectively. Both objectives have favorable properties for fitting a proposal distribution~\cite{Chatterjee2018,Agapiou2017}. IWVI~\cite{NIPS2018_7699} seeks to maximize a tight lower bound of the evidence that is based on importance sampling estimates (IWELBO). IWVI empirically improves over VI for estimating posterior expectations. It is unclear, however, which method to choose to address a particular problem.

Second, even if we can faithfully compute expectations of the model posterior, the model learned by the VAE may not resemble the real data-generating process~\cite{Turner2011}. Most VAE frameworks rely on the IWELBO, where the variational distribution is used as a proposal~\cite{BurdaGS15,rainforth2018tighter,chen2018variational}. For example, model and inference parameters are jointly learned in the IWAE~\cite{BurdaGS15} using the IWELBO. Similarly, the wake-wake (WW) procedure~\cite{Bornschein2015,le2018revisiting} uses the IWELBO for learning the model parameters but seeks to find a variational distribution that minimizes the forward KL divergence. 
% MJ: Not clear how this contributes to the argument.
%For the sake of completeness, we also investigate the $\rchi$-VAE, a novel variant of the WW algorithm that for fixed $p_\theta$ minimizes the $\rchi^2$ divergence (Appendix~\ref{app:chi-vaes}). %(instead of the forward KL), estimated with the variational upper-bound CUBO~\cite{NIPS2017_6866} 
%To avoid cases where the $\rchi^2$ divergence is infinite, we may use a Student-t variational distribution.
% In particular, we cannot ensure the proposal used for model learning is desirable for decision-making.

To address both of these issues, we propose a simple three-step procedure for making decisions with VAEs . First, we fit a model based on one of several objective functions (e.g., VAE, IWAE, WW or $\rchi$-VAE) and select the best model based on some metric (e.g., IWELBO calculated on held-out data with a large numbers of particles). The $\rchi$-VAE is a novel variant of the WW algorithm that, for fixed $p_\theta$, minimizes the $\rchi^2$ divergence (further details in Appendix~\ref{app:chi-vaes}). %(instead of the forward KL), estimated with the variational upper-bound CUBO~\cite{NIPS2017_6866} 
%To avoid cases where the $\rchi^2$ divergence is infinite, we may use a Student-t variational distribution.
Second, with the model fixed, we fit several approximate posteriors, based on the same objective functions, as well as annealed importance sampling~\cite{wu2016quantitative} where applicable. Third, we combine the approximate posteriors as proposal distributions for multiple importance sampling~\cite{veach1995optimally} to make decisions that minimize the expected loss under the posterior. 
% For this last step, selecting the best proposal may be a hard problem (e.g., it may depend on the generative model or the shape of the candidate variational distributions). We therefore propose to fit multiple variational distributions and combine them using multiple importance sampling.
% \jeff{Could we just always say to use multiple IS the last step? That would be more concrete. For the experiments, can we make it so multiple IS always comes out ahead?}
%More precisely, from a list of proposal $(q_1, \ldots, q_p)$ and coefficients $(\alpha_1, \ldots, \alpha_p)$, we create a mixture distribution $\bar{q}$ defined as 
% \(
%     \bar{q} = \sum_{i=1}^p \alpha_i q_i.
% \)
In multiple importance sampling, we expect the mixture to be a better proposal than either of its components alone, especially in settings where the posterior is complex because each component helps with different parts of the posterior.

After introducing the necessary background (Section~\ref{sec:related}),
we provide a complete analysis of our framework in setting of the probabilistic PCA model~\cite{Bishop:2006:PRM:1162264} (Section~\ref{sec:lin_VAE_analysis}). 
In this tractable setting, we recover the known fact that an underdispersed proposal causes severe error to importance sampling estimators~\cite{pmlr-v80-yao18a}. The analysis also shows that overdispersion may harm the process of model learning by exacerbating existing biases in variational Bayes. We also confirm these results empirically. Next, we perform extensive empirical evaluation of two real-world decision-making problems. 
First, we consider a practical instance of classification-based decision
theory. 
In this setting, we show that the vanilla VAE becomes overconfident in its posterior predictive density, which harms performance. We also show that our three-step procedure outperforms IWAE and WW (Section~\ref{label_decision}).
We then present a scientific case study, focusing on an instance of multiple hypothesis testing in single-cell RNA sequencing data. Our approach yields a better calibrated estimate of the expected posterior false discovery rate (FDR) than that computed by the current state-of-the-art method (Section~\ref{FDR}). The code to reproduce the experiments in this manuscript is available on GitHub \url{https://github.com/PierreBoyeau/decision-making-vaes}.