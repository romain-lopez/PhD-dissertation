With the increase of experimental protocols that generate large amounts of data~\cite{Efremova2020} (such as sequencing, microscopy) and the accumulation of large data repositories (e.g. of medical records), we expect DGMs to find numerous new applications and challenges. Indeed, Deep Generative Models bring to the table both remarkably flexible modeling capabilities and convenient inference procedures. These advantages in practice resulted in promising range of applications in molecular biology. 

We have highlighted a number of success stories in the field of single-cell transcriptomics. We also mentioned a few significant areas for improvement such as interpretability of the models, looking for causal relationships and diagnosing the quality of the inference procedures. 

Finally, code libraries are available that already implement many of the operations needed by DGM, while making use of modern computational tools such as stochastic optimization, automatic differentiation and GPU-accelerated computing. For these reasons and given the scale and complexity of current data sets (e.g., ~\cite{regev2017science,davis2018encyclopedia,bento2014chembl}), DGMs may become an integral part of the standard analysis toolbox in the life sciences.




