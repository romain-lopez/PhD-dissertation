\chapter{Introduction}
This introduction provides an overview of the field of single-cell genomics, as well as context for each chapter of this dissertation. We also briefly present supplementary contributions from my doctoral studies that are not exposed in this dissertation. 

\section{Brief exposition to cellular heterogeneity}

Cell is the basic unit of life. Developing better understanding of cellular function and states in tissue promises to unravel the most intriguing secrets of developmental biology, but also to apprehend the most complicated mechanics of the immune system. This is a difficult task, because there are many independent factors that concurrently shape cellular heterogeneity, inside of a single organism~\cite{wagner.revealing}. For example, immune cells may respond to a stimulus and produce various types of cytokines. Cells may also interact with surrounding cells and therefore are heavily impacted by their spatial context. Researchers have therefore built several single-cell experimental assays, each capturing a unique view on cellular heterogeneity. 

\section{The single-cell omics era}

The variety of molecular interactions happening in a single cell is fascinating, although vertiginous for the neophyte. Cells have a common genetic material (DNA), that get transcribed into messenger RNA (mRNA). This process of transcription is usually conditioned by different factors such as chromatic accessibility, or DNA methylation. Once transcribed, mRNA may be translated into proteins. Proteins then take part in chemical reactions (metabolism) to affect the concentration of many chemical constituents inside of that cell. 

In the recent years, we have witnessed the developments of many methods for measuring the output of many of these mechanisms at the single-cell level (e.g., scDNA-seq~\cite{wang2014clonal}, scRNA-seq~\cite{dropseq}). Many assays recently permitted to measure two or more of those omics as multiple modalities of the same dataset such as scNMT-seq~\cite{clark2018scnmt} for transcriptomics and DNA methylation and CITE-seq~\cite{stoeckius2017simultaneous} for proteomics and transcriptomics. The content in this thesis focuses solely on scRNA-seq, although some extensions of the work have been applied to other modalities.

Another impressive area of progress in the field of single-cell omics is the throughput. In the very early stages of single-cell sequencing (2012), measuring the transcriptome of twelve cells was a clear breakthrough~\cite{ramskold2012full}. However, 10x Genomics released a dataset of gene expression for more than one million cells, a couple months after I started graduate school (2017).

As a consequence of the success of single-cell RNA sequencing, many consortia were formed to measure the transcriptome of a whole organism, as well as population. For example, Tabula Muris is a compendium of single cell transcriptome data from the model organism Mus musculus, containing nearly 100,000 cells from 20 organs and tissues~\cite{tabula2018single}. This created an exciting set of opportunities for statisticians to create scalable, and principled tools for data fusion (also referred to as data harmonization in this thesis). This is also the also the context for this thesis work.

\section{Dissertation overview}

 I first present some background on deep generative models (Chapter~\ref{background}) and in particular on auto-encoding variational Bayes~\cite{kingma.auto}. This chapter is a lightly edited version of the review paper I co-authored about deep generative models for molecular biology, and that appears in Molecular Systems Biology~\cite{lopez2020enhancing}.

The rest of this dissertation has been written in two parts. First, I introduce two algorithms we conceived for integrating massive single-cell transcriptomics datasets (scRNA-seq), and that are based on variational auto-encoders (Part~\ref{part1}). The first one is single-cell Variational Inference (scVI), exposed in Chapter~\ref{scvi}. scVI is a fully probabilistic approach for the normalization and analysis of scRNA-seq data. scVI is based on a hierarchical Bayesian model with conditional distributions specified by deep neural networks and that can be trained very efficiently even for very large datasets. This chapter has been slightly edited from the original publication that appears in Nature Methods~\cite{scvi} (early versions of this work also appeared in two workshop proceedings~\cite{MLCB,baylearn}). As I write this thesis, I have gather several biological studies where scVI was used for making scientific discoveries, including the identification of the drivers of metastasis in cancer xenografts~\cite{quinn2021single}, as well as the characterization of cellular heterogeneity for COVID-19 patients~\cite{ballestar2020single}. I additionally present single-cell ANnotation using Variational Inference (scANVI) in Chapter~\ref{scanvi}. scANVI is a semi‐supervised variant of scVI designed to leverage existing cell state annotations. This flexible framework of semi‐supervised learning can be applied to two main variants of the annotation problem. In the first scenario, we are concerned with a single dataset in which only a subset of cells can be confidently labeled (e.g., based on expression of marker genes) and annotations should then be transferred to other cells, when applicable. In the second scenario, annotated datasets are harmonized with unannotated datasets and then used to assign labels to the unannotated cells. scANVI has been applied in the context of the Tabula Sapiens study, a massive consortia that measured the transcriptome of many organs across many human donors. This chapter has been slightly edited from the form it appears in Molecular Systems Biology~\cite{Xu2019}. 

In Part~\ref{part2}, I expose two fundamental failure modes of variational auto-encoders and propose general approaches to alleviate those problems. Interestingly, I believe that approaching VAEs from an applied perspective helped me frame those problems in a unique way. Consequently, all the chapters in Part~\ref{part2} also have a short application to scRNA-seq data. In Chapter~\ref{hcv}, I expose a fundamental problem with auto-encoding variational Bayes: while providing appealing flexibility, VAEs make it difficult to impose or assess structural constraints such as conditional independence. I propose a framework for learning representations that still relies on auto-encoding variational Bayes, but in which the search space is constrained via kernel-based measures of independence (the d-variable Hilbert-Schmidt Independence Criterion) to enforce independence between the latent representations and
arbitrary nuisance factors. For broader context, I encountered this problem while trying to factor sequencing errors out of the latent representation of scVI. This chapter appeared (in a shortened version) in the proceedings of Neural Information Processing Systems~\cite{lopez.information}. In Chapter~\ref{decision}, I expose the poor performance of letting the variational distribution serve as a surrogate for the posterior distribution in the context of Bayesian decision-making. I explore how fitting the variational distribution based on several objective functions other than the ELBO, while continuing to fit the generative model based on the ELBO, affects the quality of downstream decisions. Theoretical results suggest that a posterior approximation distinct from the variational distribution should be used for making decisions. I therefore propose learning several approximate proposals for the best model and combining them using multiple importance sampling for decision-making. This work started as a quest to diagnose the performance of scVI for Bayesian differential expression. A shorted version of this chapter appeared in the proceedings of Neural Information Processing Systems~\cite{romain.decision}.

Finally, I discuss in Chapter~\ref{perspectives} several avenues for future work. This chapter is also an excerpt from the review paper cited above. 

\section{Related contributions}
Although the results presented in this dissertation constitute the core of my doctoral work, I would like to give here a quick highlight on research that I participated on as a collaborator, and that builds upon scVI. A star next to the title indicates I (co-)led the study.

\subsection[Multi-modal data analysis]{Multi-modal data analysis: transcriptomics, proteomics and spatial covariates}
I have always been interested in adding as much information as possible into the same methods, with the ultimate goal of building latent variable models that can relate or contrast modalities from data, as well as improve \textit{in-silico} the resolution of one modality (e.g., current spatial transcriptomics assays).

\paragraph{Joint probabilistic modeling of single-cell multi-omic data with totalVI} The paired measurement of RNA and surface proteins in single cells with cellular indexing of transcriptomes and epitopes by sequencing (CITE-seq)~\cite{stoeckius.simultaneous} is a promising approach to connect transcriptional variation with cell phenotypes and functions. However, combining these paired views into a unified representation of cell state is made challenging by the unique technical characteristics of each measurement. Total Variational Inference (totalVI) is a framework for end-to-end joint analysis of CITE-seq data that probabilistically represents the data as a composite of biological and technical factors, including protein background and batch effects~\cite{totalvi}. TotalVI provides a cohesive solution for common analysis tasks such as dimensionality reduction, the integration of datasets with different measured proteins, estimation of correlations between molecules and differential expression testing.

\paragraph{A joint model of unpaired data from scRNA-seq and spatial transcriptomics for imputing missing gene expression measurements$^*$}
Spatial studies of transcriptome provide biologists with gene expression maps of heterogeneous and complex tissues. However, most experimental protocols for spatial transcriptomics suffer from the need to select beforehand a small fraction of genes to be quantified over the entire transcriptome. Standard single-cell RNA sequencing (scRNA-seq) is more prevalent, easier to implement and can in principle capture any gene but cannot recover the spatial location of the cells. We therefore focused on the problem of imputation of missing genes in spatial transcriptomic data based on (unpaired) standard scRNA-seq data from the same biological tissue. Building upon domain adaptation work, we propose gimVI~\cite{lopez.joint}, a deep generative model for the integration of spatial transcriptomic data and scRNA-seq data that can be used to impute missing genes. 

\paragraph{Multi-resolution deconvolution of spatial transcriptomics data reveals continuous patterns of inflammation$^*$}
Spatial transcriptomics analysis is a promising approach towards disentangling biological intrinsic effects with environmental effects that define cellular identities. In particular, bulk-RNA spatial measurements (e.g., 10x Visium) are becoming widely available, but still do not provide single-cell resolution. Therefore, such a spatial dataset is often matched with a single-cell RNA sequencing dataset from the same tissue. In this setting, deconvolution methods first learn a dictionary of cell states from the single-cell data and then estimate the cell type proportions for the spatial transcriptomics data. These algorithms encountered significant success in tissues with discrete cell types but are more challenging in other settings for which the cell states may better be described as continuous variations within discrete cell types. To address this limitation, we present Deconvolution of Spatial Transcriptomics profiles using Variational Inference (DestVI), a Bayesian model for multi-resolution deconvolution of cell types in spatial transcriptomics data. DestVI first learns a cell type specific embedding on the single-cell data, using a deep generative model and then estimates cell type proportions along with the expression of every cell type in every spot. 


\subsection{Applications of Bayesian decision-making to single-cell transcriptomics}
Building flexible generative models is an important research topic. However, goodness of fit is a poorly informative metric with a particular problem at hand. An excellent instance of this is when the generative model is employed for Bayesian decision-making (especially for making scientific discoveries). 

\paragraph{Detecting Zero-Inflated Genes in Single-Cell Transcriptomics Data}
In single-cell RNA sequencing data, biological processes or technical factors may induce an overabundance of zero measurements. Existing probabilistic approaches to interpreting these data either model all genes as zero-inflated, or none. But the overabundance of zeros might be gene-specific. Hence, we propose the AutoZI model~\cite{clivio.detecting}, which, for each gene, places a spike-and-slab prior on a mixture assignment between a negative binomial (NB) component and a zero-inflated negative binomial (ZINB) component. We approximate the posterior distribution under this model using variational inference, and employ Bayesian decision theory to decide whether each gene is zero-inflated. 

\paragraph{An Empirical Bayes Method for Differential Expression Analysis of Single Cells with Deep Generative Models} Detecting Detecting differentially expressed genes is important for characterizing subpopulations of cells. In scRNA-seq data, however, nuisance variation due to technical factors like sequencing depth and RNA capture efficiency obscures the underlying biological signal. Deep generative models have been extensively applied to scRNA-seq data, with a special focus on embedding cells into a low-dimensional latent space and correcting for batch effects. However, little attention has been given to the problem of utilizing the uncertainty from the deep generative model for differential expression. Furthermore the existing approaches do not allow to control for the effect size or the false discovery rate. Here, we present lvm-DE, a generic Bayesian approach for performing differential expression from using a fitted deep generative model, while controlling the false discovery rate. We apply the lvm-DE framework to scVI and scSphere, two deep generative models. The resulting approaches outperform the state-of-the-art methods at estimating the log-fold-change in gene expression levels, as well as detecting differentially expressed genes between subpopulations of cells.

\subsection{Open-source software development}

Most of the work presented in this thesis builds off the \texttt{scvi} codebase, built in the summer of the year 2018. This codebase was initially meant for external use by the single-cell community, as well as internal use for creating new models in the scVI team. More recently, we envisioned to make this prototyping tool more widely available. 

\paragraph{A library for deep probabilistic analysis of single-cell omics data$^*$}

Probabilistic models have demonstrated state-of-the-art performance for many single-cell omics data analysis tasks, including dimensionality reduction, clustering, differential expression, annotation, and removal of unwanted variation.
As many of these models use scalable stochastic inference techniques, they will also be critically important in light of growing single-cell dataset sizes.
However, the adoption of probabilistic models in data analysis pipelines is hindered by a fractured software ecosystem resulting in an array of packages with distinct, and often complex interfaces.
To address this issue, we developed \texttt{scvi-tools} (\url{https://scvi-tools.org}), a Python package that includes user-friendly implementations of ten leading probabilistic methods that can also interface with Scanpy, Seurat, and Bioconductor workflows. 

\subsection{Counterfactual inference from observational data}

Causal inference is relevant to many scientific applications, including for understanding gene regulation. However, it is also relevant for internet marketing, when one would like to evaluate the performance of an algorithm deployed online based on observational data. I have worked on two of those problems, while interning at Ant Financial as well as Amazon. I had hoped to apply a subset of these tools to single-cell biology (on pooled CRISPR screenings with scRNA-seq readouts) during my PhD, but this is left as future work. 

\paragraph{Cost-Effective Incentive Allocation via Structured Counterfactual Inference$^*$} We address a practical problem ubiquitous in modern industry, in which a mediator tries to learn a policy for allocating strategic financial incentives for customers in a marketing campaign and observes only bandit feedback. In contrast to traditional policy optimization frameworks, we rely on a specific assumption for the reward structure and we incorporate budget constraints. We develop a new two-step method for solving this constrained counterfactual policy optimization problem. First, we cast the reward estimation problem as a domain adaptation problem with supplementary structure. Subsequently, the estimators are used for optimizing the policy with constraints. We establish theoretical error bounds for our estimation procedure and we empirically show that the approach leads to significant improvement on both synthetic and real datasets.

\paragraph{Learning from eXtreme Bandit Feedback$^*$}
We study the problem of batch learning from bandit feedback in the setting of extremely large action spaces. Learning from extreme bandit feedback is ubiquitous in recommendation systems, in which billions of decisions are made over millions of choices in a single day, yielding massive observational data. In these large-scale real-world applications, supervised learning approaches such as eXtreme Multi-label Classification (XMC) remain the standard approach despite the bias inherent in the data collection process. Conversely, previously developed importance sampling approaches are unbiased but suffer from impractical variance when dealing with a large number of actions. In this paper, we introduce a selective importance sampling estimator (sIS) with more favorable bias-variance tradeoff. We employ this estimator in a novel algorithmic procedure—named Policy Optimization for eXtreme Models (POXM)—for learning from bandit feedback on XMC tasks. 





